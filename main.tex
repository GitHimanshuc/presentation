\documentclass[hyperref={bookmarks=false},aspectratio=169]{beamer}
\usepackage[utf8]{inputenc}
\usepackage{amsmath}
\usepackage{graphicx}
\usepackage{subcaption}
\usepackage{array}
% ---------------  Define theme and color scheme  -----------------
\usetheme[minimal]{Caltech}  % 3 options: minimal, sidebarleft, sidebarright

%\setbeamertemplate{footline}[frame number]

% ------------  Information on the title page  --------------------
\title[A Swampland Constraint on Gravitational Collapse]
{\bfseries{A Swampland Constraint on Gravitational Collapse}}

\subtitle{Master's thesis by Himanshu Chaudhary}

\author[Himanshu Chaudhary]
{\textbf{Advisor}: Chethan Krishnan\inst{1}
\and \\
\textbf{Committe members}:
Prasad Hegde\inst{1} \and Sachindeo Vaidya\inst{1}
} 
% {Mischief\inst{1} \and Managed\inst{2}}

\institute[IISc]
{
  \inst{1}
  Center for High Energy Physics\\
  Indian Institute of Science
  % \and
  % \inst{2}
  % Department of Student Pranks\\
  % California Institute of Technology
}

\date[IISc, 2020]
{1st July, 2020}
% {International Conference on University Pranks\\April 1st, 2014}
%------------------------------------------------------------

%------------------------------------------------------------
%The next block of commands puts the table of contents at the 
%beginning of each section and highlights the current section:

% \AtBeginSection[]
% {
%   \begin{frame}
%     \frametitle{Table of Contents}
%     \tableofcontents[currentsection]
%   \end{frame}
% }
%------------------------------------------------------------


\begin{document}

\frame{\titlepage}  % Creates title page

% %---------   table of contents after title page  ------------
% \begin{frame}
%   \frametitle{Table of Contents}
%   % \tableofcontents
% \end{frame}
% %---------------------------------------------------------


% \section{Caltech student traditions}

%---------------------------------------------------------
%Changing visivility of the text
\begin{frame}
    \frametitle{Introduction}

    The main focus of the thesis was to study the gravitational collapse of a scalar field and analyze the results in context of the Swampland distance bounds.

    To study the collapse Finite difference methods were used.

    In this presentation we will briefly cover the background required to appreciate the results and describe the methods use to get those results.


    % \begin{itemize}
    %   \item<1-> A claim was once made that the shattering of a pumpkin frozen in liquid nitrogen and dropped from a sufficient height would produce a triboluminescent spark.
    %   \item<2-> This yearly event involves a crowd of observers, who try to spot the elusive spark.
    %   \item<3-> The title of the event is an oblique reference to the famous Millikan oil-drop experiment which measured $e$, the elemental unit of electrical charge.
    % \end{itemize}

\end{frame}





\begin{frame}
    \frametitle{Effective field theories}
    At its core and EFT is an approximate of the underlying physical or more complete theory.
    All EFTs are valid only for some range of energies, one can not trust the results given by an EFT outside this range.

    To get and EFT at a particular energy scale one removes those degrees of freedom that are not relevant at that energy scale. Some example of EFTs used in Physics are Fermi theory of beta decay and BCS theory of superconductivity.

    \begin{alertblock}{}
        General relativity is believed to be an EFT of some underlying quantum gravity theory.
    \end{alertblock}

\end{frame}


\begin{frame}
    \frametitle{Swampland Distance Conjecture}

    \begin{block}{Landscape}
        Low energy EFTs that can be UV completed into a theory of quantum gravity.
    \end{block}
    \begin{block}{Swampland}
        Consistent looking low energy theories that lead to inconsistencies if one tries to UV complete them into a theory of quantum gravity.
    \end{block}

    Swampland Conjectures are a set of criteria that tells us whether and EFT belongs to the Landscape or the Swampland. We are particularly interested in the Swampland distance conjecture (SDC).
    \begin{alertblock}{Swampland Distance Conjecture}
        An EFT that has gravity and scalar fields cannot be reliable in the regimes where the scalar field moves beyond an $O(1)$ range.
    \end{alertblock}

\end{frame}


\begin{frame}
    \frametitle{SDC and gravitational collapse}

    We studied the evolution of a scalar field minimally coupled to gravity.
    \begin{equation}
        S=\frac{1}{8 \pi G} \int d^{4} x \sqrt{-g}\left(\frac{1}{2} R-\frac{1}{2} \partial_{\mu} \phi \partial^{\mu} \phi\right)
        \label{eqn:action}
    \end{equation}

    \begin{alertblock}{}
        Our results show that the scalar field moves by $O(1)$ range and this results seems to be independent of the initial profile of the scalar field.
    \end{alertblock}

    This results shows the sharp tension between the SDC and the belief that EFT should be well defined at the horizon.


\end{frame}

% \begin{frame}
%     \frametitle{Gravitational Collapse}

%     \begin{itemize}
%         \item Describe collapse of a minimally coupled scalar field.
%         \item Talk about the known results
%         \item Talk about our results
%     \end{itemize}

% \end{frame}


% \begin{frame}
%     \frametitle{Scalar gravitational collapse}
%     % \begin{itemize}
%     %   \item Write the action
%     %   \item The metric
%     %   \item The
%     % \end{itemize}
%     \begin{equation*}
%         S=\frac{1}{8 \pi G} \int d^{4} x \sqrt{-g}\left(\frac{1}{2} R-\frac{1}{2} \partial_{\mu} \phi \partial^{\mu} \phi\right)
%     \end{equation*}
%     \begin{equation*}
%         ds^2=e^{-2 \sigma(t, x)}\left(-d t^{2}+d x^{2}\right)+r^{2}(t, x) d \Omega^{2}
%     \end{equation*}
%     \begin{equation*}
%         r\left(-r_{, t t}+r_{, x x}\right)-r_{, t}^{2}+r_{x}^{2}=e^{-2 \sigma}
%     \end{equation*}

%     \begin{equation*}
%         -\sigma_{, t t}+\sigma_{,x x}+\frac{r_{, t t}-r_{,x x}}{r}+4 \pi\left(\psi_{, t}^{2}-\psi_{, x}^{2}\right)=0
%     \end{equation*}

%     \begin{equation*}
%         -\psi_{, t t}+\psi_{,x x}+\frac{2}{r}\left(-r_{, t} \psi_{, t}+r_{, x} \psi_{, x}\right)=0
%     \end{equation*}


% \end{frame}

% \begin{frame}
%     \frametitle{Constraint equations}
%     \begin{equation}
%         r_{, t x}+r_{, t} \sigma_{, x}+r_{, x} \sigma_{, t}+4 \pi r \psi_{, t} \psi_{, x}=0
%         \label{eqn:constraint_1}
%     \end{equation}

%     \begin{equation}
%         r_{, t t}+r_{, x x}+2 r_{, t} \sigma_{, t}+2 r_{, x} \sigma_{, x}+4 \pi r\left(\psi_{, t}^{2}+\psi_{, x}^{2}\right)=0
%         \label{eqn:constraint_2}
%     \end{equation}

%     They will be used to generate initial conditions

%     They will be used to check out results



% \end{frame}

\begin{frame}
    \frametitle{Simulating the gravitational collapse}

    The metric is of the form:
    \begin{equation*}
        ds^2=e^{-2 \sigma(t, x)}\left(-d t^{2}+d x^{2}\right)+r^{2}(t, x) d \Omega^{2}
    \end{equation*}

    The equations that we get from the action (\ref{eqn:action}) are not fit for numerical evolution as they lead to instabilities. To get a numerically stable system we introduce Meissner-Sharp mass $m$ defined by:

    \begin{equation}
        g^{\mu \nu} r_{, \mu} r_{, \nu}=e^{2 \sigma}\left(-r_{, t}^{2}+r_{, x}^{2}\right) \equiv 1-\frac{2 m}{r}
        \label{eqn:m_def}
    \end{equation}

\end{frame}


\begin{frame}
    \frametitle{Final equations to solve}

    By using the definition of $m$ we get the following set of 4 equations that we need to simulate.
    \begin{equation}
        -\psi_{, t t}+\psi_{, x x}+\frac{2}{r}\left(-r_{, t} \psi_{, t}+r_{, x} \psi_{, x}\right)=0
        \label{eqn:psi}
    \end{equation}

    \begin{equation}
        -r_{, t t}+r_{, x x}-e^{-2 \sigma} \cdot \frac{2 m}{r^{2}}=0
        \label{eqn:r}
    \end{equation}

    \begin{equation}
        -\sigma_{, t t}+\sigma_{, x x}-e^{-2 \sigma} \cdot \frac{2 m}{r^{3}}+4 \pi\left(\psi_{, t}^{2}-\psi_{, x}^{2}\right)=0
        \label{eqn:sigma}
    \end{equation}

    \begin{equation}
        m_{, t}=4 \pi r^{2} \cdot e^{2 \sigma}\left[-\frac{1}{2} r_{, t}\left(\psi_{, t}^{2}+\psi_{, x}^{2}\right)+r_{, x} \psi_{, t} \psi_{, x}\right]
        \label{eqn:m_t}
    \end{equation}

    In addition to these four equations there are two constraint equations as well, which are used to get the initial data and to check the validity of our results.


\end{frame}

% \begin{frame}
%     \frametitle{Boundary conditions}
%     To get the boundary conditions at the origin we will use regularity arguments.
%     Observe that $r$ is always set to $0$ at $x=0$, this is done to prevent formation of any kind of cusp at the origin, which give us $r_{,t}(t,0) = 0$ and $r_{,tt}(t,0)=0$.
%     Now, to ensure that the term $\frac{2}{r}\left(-r_{, t} \psi_{, t}+r_{, x} \psi_{, x}\right)$ from the equation \ref{eqn:psi} is regular at the origin we need that $r_{, x} \psi_{, x} = 0$ because $r_{,t}(t,0)$ is already $0$, therefore we have that $\psi_{,x}(t,0) = 0$. Looking at the equation \ref{eqn:m_definition} we can see that we need $m(t,0) = 0 $ to keep the term $\frac{2 m}{r}$ regular. Similarly from the equation \ref{eqn:r} we get that $r_{,xx}=0$. Finally, at the origin the equation \ref{eqn:constraint_2} becomes $r_{,x} \sigma_{,x} = 0$ which gives us $\sigma_{,x}(t,0) = 0$.

% \end{frame}


\begin{frame}
    \frametitle{Boundary conditions and Initial conditions}
    To evolve the equations we need spatial boundary conditions which we get by using the fact that $r=0$ at the origin and regularity arguments.
    \begin{eqnarray*}
        r(t,0) = 0 ;\;\; m(t,0) =0 ;\;\; \psi_{,x}(t,0) = 0 ;\;\; \sigma_{,x}(t,0) = 0
    \end{eqnarray*}

    We also need the initial data for our system, to get that we will impose:
    \begin{equation*}
        r_{, t}=\sigma_{, t}=\phi_{, t}=\psi_{, t}=0 \quad \text { at } t=0
        \label{eqn:time_symmetric_boundary_conditinons}
    \end{equation*}
    Using these conditions and the equations we derived earlier we get the two ODEs that describe the spatial variation of $r$ and $\sigma$. To get the ODE that describes the spatial variation of $m$ we differentiate the equation(\ref{eqn:m_def}) w.r.t $x$. Now, we only need the initial profile $\psi(x)$ which is what we give as the input.
\end{frame}


\begin{frame}
    \frametitle{Inital conditions}

    System of ODEs that define our initial conditions:

    \begin{equation}
        r_{, x x}=e^{-2 \sigma} \cdot \frac{2 m}{r^{2}}
        \label{eqn:r_chap3}
    \end{equation}


    \begin{equation}
        \sigma_{, x}= -2 \pi \cdot  \frac{\psi_{, x}^{2} \cdot r}{r_{,x}}- e^{-2 \sigma} \cdot \frac{ m}{r^{2}r_{, x}}
        \label{eqn:sigma_chap3}
    \end{equation}

    \begin{equation}
        m_{, x}=4 \pi r^{2} \cdot e^{2 \sigma}\left[\frac{1}{2} r_{, x} \cdot \psi_{, x}^{2} \right]
    \end{equation}
    Now, we need boundary conditions for these which we take to be:
    \begin{eqnarray*}
        r(0 ,0) = 0  ;\;\;
        r_{,x}(0 ,0) = 1  ;\;\;
        \sigma(0 ,0) = 1  ;\;\;
        m(0 ,0) = 1
    \end{eqnarray*}

\end{frame}


% \begin{frame}
%     \frametitle{Boundary Condtitinos for the inital conditions}
%     \begin{eqnarray*}
%         r(0 ,0) = 0  \\
%         r_{,x}(0 ,0) = 1  \\
%         \sigma(0 ,0) = 1  \\
%         m(0 ,0) = 1
%     \end{eqnarray*}
% \end{frame}

% \begin{frame}
%     \frametitle{Floating point errors}
%     What are floating point errors?
%     \begin{table}[hbt!]
%         \centering
%         \begin{tabular}{||m{2cm} | m{3.0cm} | m{1.5cm} | m{4.5cm}||}
%             \hline
%             Float type & Largest Number that can be stored* & Precision & How is $\frac{1}{3}$ internally stored \\ [0.5ex]
%             \hline\hline

%             Float32    & 3.40e+38                           & 6         & 0.33333334                             \\

%             Float64    & 1.79e+308                          & 15        & 0.3333333333333333                     \\

%             Float128   & 3.36e+4932                         & 18        & 0.33333333333333333334                 \\ [1ex]
%             \hline
%         \end{tabular}
%         \caption{This table shows the largest number that can be represented by a particular type of float (* rounded off to two decimal places). Precision denotes the number of significant decimal digits that can be represented by a float type.}
%         \label{table:floats}
%     \end{table}
% \end{frame}

% \begin{frame}
%     \frametitle{Finite difference: First Order}
%     \begin{columns}
%         \column{0.3\textwidth}
%         \begin{equation*}
%             y'(x_0)  \approx \frac{y(x_0 + h) - y(x_0)}{h}
%             \label{eq:1d_1o_error}
%         \end{equation*}
%         \column{0.7\textwidth}
%         \begin{figure}[hbt!]
%             \centering
%             \includegraphics[width=\textwidth]{images/x^3_error_order1.png}
%             % \caption[Error in the approximation of the first derivative of $x^3$ by first order finite difference methods.]{This plot show error of using first order finite difference to approximate the derivative of $x^3$ vs the step size. Observe that the error falls almost linearly until the floating point errors start to dominate, after which it starts to grow erratically. Also, observe that for reasonable step sizes the error falls with a slope of 1, which is why such approximations are called to be of first order. }\label{fig:x^3_error_order1}
%             \index{figures}


%         \end{figure}

%     \end{columns}
% \end{frame}


\begin{frame}
    \frametitle{Finite difference(FD) methods }

    \begin{columns}
        \column{0.35\textwidth}
        We will be using second order FD to solve our equations.
        Using the notation $X^{n}_{j}  = X(n \delta t, j \delta x)$ the FD approximations are:
        \begin{eqnarray*}
            X^n_{j,t} &=& \frac{X^{n+1}_{j} - X^{n-1}_{j}}{2 \delta t} \\
            X^n_{j,tt} &=& \frac{X^{n+1}_{j} -2X^n_j + X^{n-1}_{j}}{\delta t^2}
            \label{eq:d_second_order}
        \end{eqnarray*}
        \column{0.7\textwidth}
        \begin{figure}[hbt!]
            \centering
            \includegraphics[width=\textwidth]{images/x^3_error_order2.png}
            % \caption[Error in the approximation of the first derivative of $x^3$ by first order finite difference methods.]{This plot show error of using first order finite difference to approximate the derivative of $x^3$ vs the step size. Observe that the error falls almost linearly until the floating point errors start to dominate, after which it starts to grow erratically. Also, observe that for reasonable step sizes the error falls with a slope of 1, which is why such approximations are called to be of first order. }\label{fig:x^3_error_order1}
            \index{figures}


        \end{figure}

    \end{columns}
\end{frame}


\begin{frame}
    \frametitle{Errors in FD for functions with large derivatives}
    \begin{columns}
        \column{0.3\textwidth}
        We used second order FD because first order FD is very inaccurate for functions with large derivatives.

        This graph also shows that higher order FD methods give diminishing returns. In addition to that they are also much harder to code.
        \column{0.7\textwidth}
        \begin{figure}[hbt!]
            \centering
            \includegraphics[width=\textwidth]{images/1_x_error_vs_order.png}

            \index{figures}
        \end{figure}
    \end{columns}

\end{frame}

\begin{frame}
    \frametitle{Stencil used}
    \begin{columns}
        \column{0.3\textwidth}
        \textbf{Green}: Spatial boundary conditions

        \textbf{Blue}: Initial conditions

        \textbf{Red}: First time step, we used taylor series

        \textbf{Yellow}: We can use our stencil

        For outer boundary we used extrapolation boundary conditions.

        \column{0.7\textwidth}
        \begin{figure}[hbt!]
            \centering
            \includegraphics[width=.75\textwidth]{images/labelled_grid.eps}

        \end{figure}
    \end{columns}

\end{frame}


\begin{frame}
    \frametitle{Results: Gaussian Profile}

    \begin{equation*}
        \psi(t=0, x)=A \exp \left(\frac{-\left(x-x_{0}\right)^{2}}{\delta^{2}}\right)
    \end{equation*}

    \begin{columns}
        \column{0.5\textwidth}
        \begin{figure}
            \centering
            \includegraphics[width=1\linewidth]{images/super_Gaussian.pdf}
        \end{figure}
        \column{0.5\textwidth}
        \begin{figure}
            \centering
            \includegraphics[width=1\linewidth]{images/at0_Gaussian.pdf}
        \end{figure}
    \end{columns}

\end{frame}

\begin{frame}
    \frametitle{Results: Maxwell Profile}

    \begin{equation*}
        \psi(t=0, x)=A x^{2} \exp \left(\frac{-\left(x-x_{0}\right)^{2}}{\delta^{2}}\right)
    \end{equation*}

    \begin{columns}
        \column{0.5\textwidth}
        \begin{figure}
            \centering
            \includegraphics[width=1\linewidth]{images/super_mod.pdf}
        \end{figure}
        \column{0.5\textwidth}
        \begin{figure}
            \centering
            \includegraphics[width=1\linewidth]{images/at0_mod.pdf}
        \end{figure}
    \end{columns}

\end{frame}

\begin{frame}
    \frametitle{Results: Tanh profile}

    \begin{equation*}
        \psi(t=0, x)=A\left(\tanh \left(\frac{-\left(x-x_{0}\right)}{\delta^{2}}\right)+1\right)
    \end{equation*}

    \begin{columns}
        \column{0.5\textwidth}
        \begin{figure}
            \centering
            \includegraphics[width=1\linewidth]{images/super_tanh.pdf}
        \end{figure}
        \column{0.5\textwidth}
        \begin{figure}
            \centering
            \includegraphics[width=1\linewidth]{images/at0_tanh.pdf}
        \end{figure}
    \end{columns}

\end{frame}

\begin{frame}
    \frametitle{Results: Shell profile}

    \begin{equation*}
        \psi(t=0, x)=A\left(\tanh \left(\frac{-\left(x-x_{0}\right)}{\delta^{2}}\right)+\tanh \left(\frac{\left(x-x_{0}+w\right)}{\delta^{2}}\right)\right)
    \end{equation*}

    \begin{columns}
        \column{0.5\textwidth}
        \begin{figure}
            \centering
            \includegraphics[width=1\linewidth]{images/super_shell.pdf}
        \end{figure}
        \column{0.5\textwidth}
        \begin{figure}
            \centering
            \includegraphics[width=1\linewidth]{images/at0_shell.pdf}
        \end{figure}
    \end{columns}

\end{frame}


\begin{frame}
    \frametitle{Conclusions}

    The results that we have show a sharp tension between SDC and the belief that EFT should hold at the horizon of a Black hole.

    Because both these ideas are still neither experimentally verified nor have a sound theoretical"proof", we hope that this extra bit of insight will give us another way of analyzing them.

\end{frame}


\begin{frame}{}
    \centering \Large
    \emph{Thank you}
\end{frame}

%---------------------------------------------------------


%---------------------------------------------------------
% \begin{frame}  % Example of the \pause command
%   This slide is to test mathematical formulas \pause

%   $$E=mc^2$$ \pause

%   as well as the ``pause'' functionality
% \end{frame}
%---------------------------------------------------------

% \section{Caltech student pranks}

% %---------------------------------------------------------
% %Highlighting text
% \begin{frame}
%   \frametitle{Gravitational Collapse}

%   This is a brief introduction of \alert{Caltech pranks}.

%   \begin{block}{Definition}
%     Prank: a practical joke or mischievous act
%   \end{block}

%   \begin{alertblock}{Axiom}
%     Caltech pranks are a key part of the institute's history and identity.
%   \end{alertblock}

%   \begin{examples}
%     See the next slide for a prank example.
%   \end{examples}
% \end{frame}
% %---------------------------------------------------------


% %---------------------------------------------------------
% %Two columns
% \begin{frame}
%   \frametitle{Hollywood sign}

%   \begin{columns}

%     \column{0.45\textwidth}

%     \begin{figure}
%       \centering
%       \includegraphics[width=\columnwidth]{./figures/hollywood_caltech.jpg}
%       \caption{``Hollywood is still mad about that,'' says Autumn Looijen, author of \emph{Legends of Caltech III: Techer In the Dark.} \tiny{(Photo downloaded from: http://brennen.caltech.edu/autobiography/automaster2.htm)}}
%       \label{fig:hollywood_prank}
%     \end{figure}


%     \column{0.55\textwidth}
%     In May 1987, undergraduates from Page and Ricketts houses combined forces (and several hundred dollars) to purchase enough black and white plastic, transformed the Hollywood sign to read ``Caltech''.

%     \small{(Reference: http://www.admissions.caltech.edu/pranks)}

%   \end{columns}
% \end{frame}
%---------------------------------------------------------


\end{document}